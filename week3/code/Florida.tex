\documentclass[12]{article}
\usepackage[utf8]{inputenc}
\usepackage{graphicx}
\usepackage{float}
\usepackage[a4paper,
            left=1in,
            right=1in,
            top=0.5in,
            bottom=1in,]{geometry}



\title{Is Florida Getting Warmer?}
\author{Anqi Wang}
\date{Oct, 2022}

\begin{document}
	\maketitle

	\section{Results}
	The Pearson correlation coefficient was used to determine whether there was a positive correlation 		between Year and Temperature in Key West, Florida. The correlation coefficient of the Temperature 		data throughout the years in the 20th century was 0.53317 (to 5 s.f.), and a permutation analysis (		Figure 1) of 1000 shuffled populations was used to determine the approximate, asymptotic, one-		tailed P-value of 0.

	\begin{figure}[H]
		\centering
		\includegraphics[scale=0.75]{../results/atsplot.pdf}
		\caption{Raw data of Temperature over time}
	\end{figure}

	\begin{figure}[H]
		\centering
		\includegraphics[scale=0.75]{../results/allcoeffs.pdf}
		\caption{Coefficients}
	\end{figure}

	\begin{figure}[H]
		\centering
		\includegraphics[scale=0.75]{../results/AsymptoticPValue.pdf}
		\caption{Histogram of Coefficients}}
	\end{figure}

	\section{Discussion}
	The null hypothesis that there is no relationship between Time and Temperature in Key West, Florida, 	USA can be reject due to the small P-value below 0.05, and instead there is a statistically significant 		positive correlation between the two variables.\\
	The positive correlation indicated between Year and Temperature in Key West is statistically 			significant. \\
	Moreover, these results suggest that the temperature was increasing over time in Key West, Florida, 	USA, in the 20th century.

\end{document}
